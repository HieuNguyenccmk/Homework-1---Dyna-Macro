\documentclass{article}
\usepackage[T1]{fontenc}
\usepackage[utf8]{inputenc}
\usepackage{subcaption}  
\usepackage{float}  

% Packages
\usepackage{amsmath, amssymb, graphicx, booktabs, hyperref, natbib, adjustbox}
\usepackage[T1]{fontenc}
\usepackage[utf8]{inputenc}
\usepackage[vietnamese,english]{babel}
\usepackage[a4paper, margin=1in]{geometry}  
\usepackage{algorithm, algorithmicx}

% Macros for section formatting
\newcommand{\sectiontitle}[1]{\section*{#1}\addcontentsline{toc}{section}{#1}}
\newcommand{\subsectiontitle}[1]{\subsection*{#1}\addcontentsline{toc}{subsection}{#1}}

\begin{document}
\begin{center}
    \textbf{Problem Set 1 - Dynamic Macroeconomics} \\[1em]
    \textbf{Nguyen Quoc Hieu} \\[0.5em]
    \textbf{22 March 2025} \\[0.5em]
    \textbf{Fulbright University Vietnam}
\end{center}

\subsectiontitle{Question 1(a): Write the time series model in matrix form}

The given time series model is:
\begin{align*}
    y_t &= \alpha_0 + \alpha_1 y_{t-1} + \alpha_2 c_{t-1} + \varepsilon_t, \\
    c_t &= \beta_0 + \beta_1 c_{t-1} + \beta_2 c_{t-2} + \beta_3 y_{t-1} + \nu_t,
\end{align*}
where the error terms are:
\[
\bm{\xi}_t = 
\begin{bmatrix}
    \varepsilon_t \\ 
    \nu_t
\end{bmatrix}
\sim \mathcal{N} \left(
\begin{bmatrix}
    0 \\ 
    0
\end{bmatrix}, 
\bm{\Sigma} =
\begin{bmatrix}
    \sigma^2_\varepsilon & 0 \\ 
    0 & \sigma^2_\nu
\end{bmatrix}
\right).
\]

\textbf{Matrix Form Representation}
Define the state vector as:
\[
\bm{z}_t = 
\begin{bmatrix}
    y_t \\ 
    c_t
\end{bmatrix}.
\]
The model can be rewritten as:
\[
\bm{z}_t = \bm{A} + \bm{B} \bm{z}_{t-1} + \bm{C} \bm{z}_{t-2} + \bm{\xi}_t,
\]
where:
\[
\bm{A} = 
\begin{bmatrix}
    \alpha_0 \\ 
    \beta_0
\end{bmatrix}, \quad
\bm{B} = 
\begin{bmatrix}
    \alpha_1 & \alpha_2 \\ 
    \beta_3 & \beta_1
\end{bmatrix}, \quad
\bm{C} = 
\begin{bmatrix}
    0 & 0 \\ 
    0 & \beta_2
\end{bmatrix}, \quad
\bm{\xi}_t = 
\begin{bmatrix}
    \varepsilon_t \\ 
    \nu_t
\end{bmatrix}.
\]

\textbf{Full Expanded System}
\[
\begin{bmatrix}
    y_t \\ 
    c_t
\end{bmatrix}
=
\begin{bmatrix}
    \alpha_0 \\ 
    \beta_0
\end{bmatrix}
+
\begin{bmatrix}
    \alpha_1 & \alpha_2 \\ 
    \beta_3 & \beta_1
\end{bmatrix}
\begin{bmatrix}
    y_{t-1} \\ 
    c_{t-1}
\end{bmatrix}
+
\begin{bmatrix}
    0 & 0 \\ 
    0 & \beta_2
\end{bmatrix}
\begin{bmatrix}
    y_{t-2} \\ 
    c_{t-2}
\end{bmatrix}
+
\begin{bmatrix}
    \varepsilon_t \\ 
    \nu_t
\end{bmatrix}.
\]

\subsection*{Question 1(b)}
\subsection*{Part (b): AR(1) Representation}

Refine the state vector to include necessary lags to convert the model to an AR(1) process.

\textbf{Define the State Vector}
\[
\bm{s}_t = 
\begin{bmatrix}
    y_t \\ 
    c_t \\ 
    c_{t-1}
\end{bmatrix},
\quad \text{with lagged state vector } \bm{s}_{t-1} = 
\begin{bmatrix}
    y_{t-1} \\ 
    c_{t-1} \\ 
    c_{t-2}
\end{bmatrix}.
\]

\textbf{Rewrite the Model}
The AR(1) matrix form is:
\[
\bm{s}_t = \bm{A} + \bm{B} \bm{s}_{t-1} + \bm{\xi}_t,
\]
where:
\[
\bm{A} = 
\begin{bmatrix}
    \alpha_0 \\ 
    \beta_0 \\ 
    0
\end{bmatrix}, \quad
\bm{B} = 
\begin{bmatrix}
    \alpha_1 & \alpha_2 & 0 \\ 
    \beta_3 & \beta_1 & \beta_2 \\ 
    0 & 1 & 0
\end{bmatrix}, \quad
\bm{\xi}_t = 
\begin{bmatrix}
    \varepsilon_t \\ 
    \nu_t \\ 
    0
\end{bmatrix}.
\]

\textbf{Final AR(1) Form}
\[
\begin{bmatrix}
    y_t \\ 
    c_t \\ 
    c_{t-1}
\end{bmatrix}
=
\begin{bmatrix}
    \alpha_0 \\ 
    \beta_0 \\ 
    0
\end{bmatrix}
+
\begin{bmatrix}
    \alpha_1 & \alpha_2 & 0 \\ 
    \beta_3 & \beta_1 & \beta_2 \\ 
    0 & 1 & 0
\end{bmatrix}
\begin{bmatrix}
    y_{t-1} \\ 
    c_{t-1} \\ 
    c_{t-2}
\end{bmatrix}
+
\begin{bmatrix}
    \varepsilon_t \\ 
    \nu_t \\ 
    0
\end{bmatrix}.
\]

\subsection{Question 2(a)}
The transition matrix $P$ for the Markov Chain is:

\[
P =
\begin{bmatrix}
0.90 & 0.10 & 0.00 \\
0.20 & 0.65 & 0.15 \\
0.10 & 0.05 & 0.85
\end{bmatrix}
\]

Each row in the matrix represents the probability of transitioning from one borrower category to another:

- The first row corresponds to Subprime borrowers.
- The second row corresponds to Average borrowers.
- The third row corresponds to Prime borrowers.

To verify the validity of the transition matrix, we check whether each row sums to 1:

\[
0.90 + 0.10 + 0.00 = 1.00
\]

\[
0.20 + 0.65 + 0.15 = 1.00
\]

\[
0.10 + 0.05 + 0.85 = 1.00
\]

Since all rows sum to 1, the transition matrix is valid.


\subsection{Question 2(b)}
I will give an example of a friend who needs a loan to help them understand their borrower category. 

Borrowers are usually categorized into three groups:
\begin{itemize}
    \item \textbf{Subprime} (risky borrowers)
    \item \textbf{Average} (medium risk)
    \item \textbf{Prime} (safe borrowers)
\end{itemize}

Every time a lender checks a borrower’s credit score, there are three possible outcomes: the borrower can be promoted to a higher category, demoted to a lower category, or remain in the same category. The probability of these transitions is as follows:

\begin{enumerate}
    \item If the borrower is in the \textbf{Subprime} category:
    \begin{itemize}
        \item 90\% chance of staying in Subprime
        \item 10\% chance of being promoted to Average
    \end{itemize}
    
    \item If the borrower is in the \textbf{Average} category:
    \begin{itemize}
        \item 65\% chance of staying in Average
        \item 15\% chance of being promoted to Prime
        \item 20\% chance of being demoted to Subprime
    \end{itemize}
    
    \item If the borrower is in the \textbf{Prime} category:
    \begin{itemize}
        \item 85\% chance of staying in Prime
        \item 5\% chance of being demoted to Subprime
        \item 10\% chance of being demoted to Average
    \end{itemize}
\end{enumerate}

All of these conditions are based on their current score or rank, meaning they are not dependent on past scores.

\subsection*{Question 3(a)}
\subsection*{Algorithm 2: Rouwenhorst's Method (Simplified)}

\textbf{Input (Parameters):} $N, p, \sigma, \mu$ \\
\textbf{Output (Grids):} $z, \pi$ \\

\textbf{Require:} $N \geq 2, 0 \leq p \leq 1$

\begin{enumerate}
    \item Compute the state space:
    \begin{align*}
        z_1 &\leftarrow \mu - \sigma \sqrt{N-1} \\
        z_N &\leftarrow \mu + \sigma \sqrt{N-1} \\
        d &\leftarrow \frac{z_N - z_1}{N-1}
    \end{align*}
    
    \item Construct the state vector:
    \begin{align*}
        \text{for } i = 1 \text{ to } N \text{ do} \quad z_i = z_1 + (i - 1) d
    \end{align*}

    \item Compute the transition matrix:
    \begin{align*}
        \text{if } N = 2, \text{ set:} \quad
        \pi_2 = 
        \begin{bmatrix}
            p & 1 - p \\
            1 - q & q
        \end{bmatrix}
    \end{align*}

    \item Recursive construction for $N > 2$:
    \begin{align*}
        \text{for } n = 3 \text{ to } N \text{ do}
    \end{align*}
    \begin{itemize}
        \item Expand $\pi_{n-1}$ using:
        \begin{align*}
            \pi_n &= p
            \begin{bmatrix}
                \pi_{n-1} & 0 \\
                0' & 0
            \end{bmatrix}
            + (1 - p)
            \begin{bmatrix}
                0 & \pi_{n-1} \\
                0' & 0
            \end{bmatrix} \\
            &+ q
            \begin{bmatrix}
                0 & 0' \\
                0 & \pi_{n-1}
            \end{bmatrix}
            + (1 - q)
            \begin{bmatrix}
                0' & 0 \\
                \pi_{n-1} & 0
            \end{bmatrix}
        \end{align*}
    \end{itemize}
    
    \item Return $\pi_N$.
\end{enumerate}

\subsection*{Algorithm 3: Rouwenhorst’s Method for AR(1)}

\textbf{Given AR(1) process:}
\begin{align*}
    y_t &= 0.5 + \gamma_1 y_{t-1} + \varepsilon_t, \quad \varepsilon_t \sim \mathcal{N}(0,1)
\end{align*}

\textbf{Input (Parameters):} $N, \gamma_1$ \\
\textbf{Output (Grids):} $y, \pi$ \\

\textbf{Require:} $N \geq 2, -1 < \gamma_1 < 1$

\begin{enumerate}
    \item Compute the state space:
    \begin{align*}
        \sigma_y &= \frac{1}{\sqrt{1 - \gamma_1^2}} \quad \text{(long-run standard deviation)} \\
        y_1 &\leftarrow 0.5 - \frac{\sigma_y \sqrt{N-1}}{1 - \gamma_1} \\
        y_N &\leftarrow 0.5 + \frac{\sigma_y \sqrt{N-1}}{1 - \gamma_1} \\
        d &\leftarrow \frac{y_N - y_1}{N-1}
    \end{align*}
    
    \item Construct the state vector:
    \begin{align*}
        \text{for } i = 1 \text{ to } N \text{ do} \quad y_i = y_1 + (i - 1) d
    \end{align*}

    \item Compute the transition probabilities:
    \begin{align*}
        p &= \frac{1 + \gamma_1}{2}, \quad q = p
    \end{align*}

    \item Compute the transition matrix:
    \begin{align*}
        \text{if } N = 2, \text{ set:} \quad
        \pi_2 = 
        \begin{bmatrix}
            p & 1 - p \\
            1 - q & q
        \end{bmatrix}
    \end{align*}

    \item Recursive construction for $N > 2$:
    \begin{align*}
        \text{for } n = 3 \text{ to } N \text{ do}
    \end{align*}
    \begin{itemize}
        \item Expand $\pi_{n-1}$ using:
        \begin{align*}
            \pi_n &= p
            \begin{bmatrix}
                \pi_{n-1} & 0 \\
                0' & 0
            \end{bmatrix}
            + (1 - p)
            \begin{bmatrix}
                0 & \pi_{n-1} \\
                0' & 0
            \end{bmatrix} \\
            &+ q
            \begin{bmatrix}
                0 & 0' \\
                0 & \pi_{n-1}
            \end{bmatrix}
            + (1 - q)
            \begin{bmatrix}
                0' & 0 \\
                \pi_{n-1} & 0
            \end{bmatrix}
        \end{align*}
    \end{itemize}
    
    \item Return $\pi_N$.
\end{enumerate}

\subsection*{Question 3(b)}
\begin{figure}[H]
    \centering
    \includegraphics[width=0.8\textwidth]{3b.png}
    \caption{State vector and transition matrix}
    \label{fig:3b}
\end{figure}

\subsection*{Question 3(c)}
\begin{figure}[H]
    \centering
    \includegraphics[width=0.8\textwidth]{3c&d.png}
    \caption{Simulated 7-State Markov Chain Using Rouwenhorst's Method}
    \label{fig:3c&d}
\end{figure}

\subsection*{Question 3(d)}
\begin{figure}[H]
    \centering
    \includegraphics[width=0.8\textwidth]{3d.png}
    \caption{Simulated 7-State Markov Chain Using Rouwenhorst's Method for different y1}
    \label{fig:3d}
\end{figure}

\subsection*{Question 4(a)}

The maintenance cost for a used tank follows a linear function:

\[
\text{Maintenance Cost} = t \cdot a(z_t)
\]

where \( t \) is the age of the tank. This implies that the maintenance cost increases proportionally with age.

The fixed cost in case of a breakdown remains:

\[
\text{Breakdown Cost} = (1 + depreciation) \cdot b(z_t)
\]
where depreciation in this case account for the damage the tank has taken.
Lastly, the cost of a new tank is given by:

\[
D(t)
\]

Since a new tank has no operational cost:

\[
c(z_t) = 0
\]

\subsection*{Question 4(b) Bellman Equation for the Military's Optimization Problem}

The military decides whether to keep or replace a tank based on its depreciation level, where depreciation \(\delta_t\) evolves from 0\% to 100\%, and a tank is replaced when \(\delta_t \geq 30\%\).

\textbf{State Variables:}
\begin{itemize}
    \item \( z_t \): Condition of the tank at time \( t \).
    \item \( \delta_t \): Depreciation level at time \( t \), where \( \delta_t \in [0,1] \).
    \item \( \varepsilon_t \): i.i.d. taste shocks affecting the decision process.
\end{itemize}

\textbf{Choice Variables:}
\begin{itemize}
    \item \textbf{Keep (\( K \))}: Continue using the tank, incurring maintenance costs \( t \cdot a(z_t) \) and breakdown costs \( (1+\delta_t) \cdot b(z_t) \).
    \item \textbf{Replace (\( R \))}: Buy a new tank at cost \( D(t) \), resetting \( \delta_t = 0 \).
\end{itemize}

\textbf{Bellman Equation:}  
Let \( V(z_t, \delta_t, \varepsilon_t) \) be the value function. The value of keeping the tank is:

\[
V_K(z_t, \delta_t, \varepsilon_t) = - t \cdot a(z_t) - (1+\delta_t) \cdot b(z_t) + \beta \mathbb{E} \left[ V(z_{t+1}, \delta_{t+1}, \varepsilon_{t+1}) \mid z_t, \delta_t \right]
\]

If the military replaces the tank when \( \delta_t \geq 30\% \), the value function for replacement is:

\[
V_R = - D(t) + \beta \mathbb{E} \left[ V(z_{t+1} = z_0, \delta_{t+1} = 0, \varepsilon_{t+1}) \right]
\]

The optimal decision follows:

\[
V(z_t, \delta_t, \varepsilon_t) = \max \{ V_K(z_t, \delta_t, \varepsilon_t), V_R \}
\]

where expectations are taken over \( \varepsilon_{t+1} \).

\subsection*{(c)Transition Probabilities for \( z_t \)}

The transition probabilities for \( z_t \) represent changes in the tank's state over time. These probabilities change based on current state variables and control choices made by the military.

\textbf{Dependence on State and Control Variables}

\textbf{State Variables}: The current state of the tank, \( z_t \), and the degree of deterioration, \( \delta_t \) (in this example, depreciation), affect the rate of change to the next state.

\textbf{Control Variables}: If the military continues to use the tank, \( z_t \) will deteriorate due to the depreciation levels \( \delta_t \) and \( t \). However, the tank will be replaced when the depreciation level is assessed to be \( \delta_t \geq 30\% \).

If the military chooses to replace the tank, the effective age will be reset to state \( z_0 \), and the maintenance cost and depreciation level will be set to \( \delta_t = 0 \).

\subsection*{Transition Probabilities}

If the military continues to use the tank, the probability distribution for \( z_{t+1} \) is conditioned on \( z_t \) and \( \delta_t \). This means that the tank's condition will gradually deteriorate over time as it accumulates wear, and eventually, it will need to be replaced.

If the tank is completely worn down, the probability of transitioning to an even worse state is zero.


\subsection*{Question 5(a)}

\subsection*{GDP}
The Covid-19 epidemic made 2020 a challenging year for the global economy. Vietnam's GDP grew by 2.91\% to \$268.4 billion, the lowest in a decade but still among the highest globally. In 2021, GDP growth was 2.58\%, 0.33\% lower than the previous year. In 2022, economic fluctuations, including the Russia-Ukraine conflict, impacted global and Vietnamese economies (Kinh te va Du bao, 2024).

\begin{figure}[H]
    \centering
    \includegraphics[width=0.8\textwidth]{Screen Shot 2025-03-09 at 16.24.31.png}
    \caption{GDP Growth in Vietnam}
    \label{fig:gdp}
\end{figure}

\FloatBarrier % Ngăn hình bị trôi xa nội dung

\subsection*{Capital formation}
From 2016 to 2020, capital mobilization through the stock market reached VND 2.67 million billion, a 133\% increase from 2011-2015. Corporate bond issuance rose by 532\%, government bonds by 54\%, and state capital divestment by 712\% (Reatimes, 2022). In 2023, total social investment capital was estimated at VND 3,423.5 trillion, up 6.2\% from 2022, with stronger investment implementation in Q4 (Vietstock, 2023).

\begin{figure}[H]
    \centering
    \includegraphics[width=0.8\textwidth]{Screen Shot 2025-03-09 at 16.55.06.png}
    \caption{Investment Growth}
    \label{fig:investment}
\end{figure}

\FloatBarrier % Ngăn hình bị trôi xa nội dung

\subsection*{Labor}
Before Covid-19, over 88\% of the population (ages 25-59) was employed (Vietnam Population and Housing Census, 2019). By September 2020, 31.8 million workers were affected, with 68.9\% experiencing income reductions, 40\% facing reduced hours, and 14\% forced to suspend work. The service sector was hit hardest (68.9\%), followed by industry/construction (66.4\%) and agriculture (27\%) (VASS, 2023).

In Q4 2021, the labor force aged 15+ was 50.7 million, up 1.7 million from the previous quarter but down 1.4 million YoY. Overall, 2021 saw a workforce decline of 791.6 thousand (Tổng cục Thống kê, 2022). In 2023, employment reached 51.3 million, increasing by 683,000 (1.35\%) from 2022 (GSO, 2024).

\begin{figure}[H]
    \centering
    \includegraphics[width=0.8\textwidth]{Screen Shot 2025-03-09 at 17.39.17.png}
    \caption{Labor Market Trends}
    \label{fig:labor}
\end{figure}

\FloatBarrier

\subsectiontitle{Question 5(B) Economic Factors Affecting Growth in Vietnam}

\subsectiontitle{Export/Import Sector}

In 2024, according to a Reuters examination of public statistics, Vietnam's goods exports to the United States amounted to 30\% of its GDP last year, the biggest share among the country's main trade partners, making it especially vulnerable to reciprocal tariffs. The tremendous entry of manufacturing investment has transformed the country into a major node in global supply networks, strengthening its economic links with the United States. Vietnam now directs 29\% of its exports to its old rival. 

According to United Nations commodities trade statistics, Vietnam became the sixth-largest exporter to the United States last year, with items valued at \$142.4 billion, following only Mexico, China, Canada, Germany, and Japan (Reuters, 2024).

In 2024, China is Vietnam's largest import supplier. According to the GSO's data breakdown, Vietnam's imports from China this year hit \$92 billion in August, a 26\% increase over the same month last year. This accounted for 37\% of Vietnam's total imports this year (Reuters, 2024).

\subsectiontitle{Policy Suggestions for Export/Import}

Vietnam should push for more free trade agreements with major partners across the world. As of October 2024, Vietnam has 17 free trade agreements (FTAs) and is currently negotiating two more with Switzerland, Norway, Iceland, Liechtenstein, and Canada (WTO, 2018).

Encouraging import and export serves as an indirect lever for other social welfare in addition to directly boosting the economy. Increased foreign investment, technological advances, a higher-quality workforce, and increased worker productivity are all benefits of expanding international commerce.

\textbf{Enhancing social welfare:} Trade expansion will increase job opportunities, improve living conditions, and lessen poverty among Vietnamese citizens.

\textbf{Benefits:} We will receive advantages including lower tax rates, cheaper transaction costs, and enhanced competitiveness for Vietnamese goods when we participate in free trade agreements (FTAs).

\subsection*{Factors from Capital (FDI, Public Investment, Government Earning and Spending)}

\subsection*{Public Investment}
In the State sector's investment capital, the capital implemented from the State budget in 2023 is anticipated to be VND 625.3 trillion, representing 85.3\% of the annual plan and a 21.2\% increase over the previous year.  
According to management level, central capital reached VND 113.5 trillion, representing 85.1\% of the yearly plan and a 24.3\% rise over the previous year, while local capital reached VND 511.8 trillion, representing 85.4\% and a 20.5\% growth.  
In locally managed capital, the provincial State budget capital reached VND 353.4 trillion, equal to 82.5\% and a 26.1\% increase; the district State budget capital reached VND 137.8 trillion, equal to 91.1\% and an 11.1\% increase; and the State budget capital at the commune level reached 20.6 trillion VND, equal to 104.7\% and a 0.2\% decrease (Ministry of Planning and Investment, 2024).

\subsection*{Foreign Direct Investment (FDI)}
The total foreign investment capital recorded in Vietnam as of December 20, 2023, which includes newly registered capital, adjusted registered capital, capital contribution, and share purchase value of foreign investors, was over 36.6 billion USD, up 32.1\% from the previous year.  
Foreign direct investment capital realized in Vietnam in 2023 is anticipated to be 23.18 billion USD, up 3.5\% over the previous year.  
This is the most realized foreign direct investment capital in the previous five years.  
The processing and manufacturing industries accounted for 19.08 billion USD, or 82.3\% of total realized foreign direct investment capital; the production and distribution of electricity, gas, hot water, steam, and air conditioning accounted for 1.37 billion USD, or 5.9\%; and real estate business activities accounted for 1.15 billion USD, or 4.9\% (Ministry of Planning and Investment, 2024).

\subsection*{Government Earnings and Spending}
In terms of state budget revenue and spending, revenue in 2023 is expected to fall by 5.4\% from the previous year.  
State budget spending is expected to rise by 10.9\% compared to 2022, meeting the demands of socioeconomic development, national defense, security, state administration, debt repayment, and prompt payment to subjects in accordance with legislation.  
Specifically, the overall state budget income in December 2023 is anticipated to be 159.6 trillion VND.  
The overall state budget income in 2023 is anticipated to be VND 1,717.8 trillion, which is 106\% of the yearly projection and a 5.4\% decrease from the previous year (Ministry of Planning and Investment, 2024).

\subsection*{Policy Suggestions for Capital Investment}
Increased investment registrations from capital sources are a positive indicator for Vietnam's economic development.  
However, the allocation of money and transparency in capital handling are equally vital.  
As a result, applying extra policy frameworks to address the situation is critical in order to avoid situations like corruption.

Resolving previous debts is also critical; being invested entails Vietnamese firms' obligation to fully follow the needed plan and repay on time.

Improving collaboration with domestic and international partners.  
Strengthening connections with neighboring nations will significantly enhance investment capital in Vietnam.  
Furthermore, private investment is a channel that requires greater attention in the growth of capital investment.  
As Vietnamese firms have begun to grow, some of which are now world leaders, drawing additional investment capital from these enterprises is also something to consider.

\subsectiontitle{Question 5(c)}

\subsection*{1. The Household}

Household preferences are determined by \( U(c, g) \), which includes private consumption, public consumption, and labor input (\( n \)).


\subsection*{Budget Constraint}

\begin{equation}
    c_t + i_t = \left(1 - \tau^k_t \right) r_t k_t + \left(1 - \tau^n_t \right) w_t n_t + \delta \tau^k_t k_t + T_t.
\end{equation}

In each period, the household uses after-tax revenue from renting out its capital and labor to buy consumption goods and invest in goods. In time \( t \), capital income is calculated as \( r_t k^*_t \), where \( r \) represents the rent price and \( k \) represents the household's capital portfolio. Labor income is \( w^*_t n_t \), where \( w^* \) represents the wage rate.

The prices of investment and consumption products are adjusted to 1 in (1). The tax rates on capital and labor income are \( \tau^k_t \) and \( \tau^n_t \), respectively. It is assumed that the tax rates are exogenous, stochastic processes. The government made \( T_t \) in lump sum transfer payments during period \( t \). Depreciation allowances \( \delta \tau^k_t k_t \), where \( \delta \in (0, 1] \), represent the constant rate of capital depreciation, providing the last source of revenue (Adda and Cooper, 2003).

\subsection*{Utility Function}

The utility function is in CRRA form:
\begin{equation}
    U(c_t, g_t) = \frac{c_t^{1-\alpha}}{1-\alpha} + \frac{g_t^{1-\alpha}}{1-\alpha}
\end{equation}

\subsection*{2. The Firms}

A Cobb-Douglas production function with a continuous return to scale controls the production.
\begin{equation}
    Y_t = A_t K_t^{\alpha} L_t^{1-\alpha}
\end{equation}
or equivalently,
\begin{equation}
    y_t = A_t k_t^{\alpha} n_t^{1-\alpha}
\end{equation}


The household's technology for converting investment and present stock into next period capital, which may be expressed as follows:
\begin{equation}
    k_{t+1} = (1 - \nu) k_t + i_t
\end{equation}

This equation represents the rule of motion for capital accumulation, where \( \delta \) is the capital depreciation rate (\( \delta \in (0,1] \)). Variations in labor productivity \( A_t \) can cause economic fluctuations or shocks. The evolution of \( A \) follows an AR(1) process:
\begin{equation}
    \log(A_{t+1}) = \mu + \rho \log(A_t) + \epsilon_{t+1}
\end{equation}

\subsection*{3. The Government}

A stochastic stream of expenditures is financed by taxes levied by the government on the rental income of sources of production. Lump-sum payments are given to families for any revenue that is not used for ongoing expenses. Thus, the actual transfers to families at time \( t \) may be found using:
\begin{equation}
    T_t = \tau^k_t r_t k_t + \tau^n_t w_t n_t - \nu \tau^k_t k_t - g_t
\end{equation}

We define \( Tt \) as the money transfer to households at first, but because gt serves the same purpose, we reduce the model to gt only.

As a result, the real transfers to households at time \( t \) are as follows:

\begin{equation}
    g_t = \tau_t^k r_t k_t + \tau_t^n w_t n_t - \nu r_t^k k_t 
\end{equation}

If we have \( g_t \) change to the left side of (2), it is essentially the budget constraint of the government. This specification assumes that the government balances its budget each period and never issues debt \cite{McGrattan1994}.

From the information above, we write the overall optimization problem:
\begin{equation}
    \max_{\{c_t\}_{t=1}^{\infty}, \{k_{t+1}\}_{t=1}^{\infty}} 
    \mathbb{E}_0 \sum_{t=1}^{\infty} \beta^t \left( \frac{c_t^{1-\alpha}}{1-\alpha} + \frac{g_t^{1-\alpha}}{1-\alpha} \right),
\end{equation}

subject to:
\begin{align}
    y_t &= c_t + i_t, \\
    y_t &= A_t k_t^\alpha, \\
    c_t + i_t &= \left( 1 - \tau_t^k \right) r_t k_t + \left( 1 - \tau_t^n \right) w_t n_t + \nu r_t^k k_t + g_t, \\
    g_t &= \tau_t^k r_t k_t + \tau_t^n w_t n_t - \nu r_t^k k_t, \\
    k_{t+1} &= (1 - \nu) k_t + i_t, \\
    \log(A_{t+1}) &= \mu + \rho \log(A_t) + \varepsilon_{t+1}, \\
    c_t &> 0, \quad k_t \geq 0, \quad n_t = 1,
\end{align}
where \( |\rho| < 1 \), \( \varepsilon_{t+1} \sim \mathcal{N}(0, \sigma_\varepsilon^2) \), and the constraints are for \( t = 1,2,\dots \).

\subsection*{Recursive Problem Formulation}

The recursive formulation, combining the constraints into one budget constraint, is:

\[
V_t(k_t, A_t) = \max_{c_t, k_{t+1}} \frac{c_t^{1-\alpha}}{1-\alpha} + \frac{(1 - g_t)^{1-\alpha}}{1-\alpha} + \beta E_t \left[ V_{t+1}(k_{t+1}, A_{t+1}) \right],
\]


\[
Y_t = A_t K_t^{\alpha} L_t^{1-\alpha}
\]

or equivalently,

\[
y_t = A_t k_t^\alpha n_t^{1-\alpha},
\]

\[
k_{t+1} = A_t k_t^\alpha n_t^{1-\alpha} - c_t + (1-\nu)k_t,
\]

\[
c_t + i_t = \left(1 - \tau^k_t \right) r_t k_t + \left(1 - \tau^n_t \right) w_t n_t + \nu \tau^k_t k_t + T_t,
\]

\[
T_t + g_t = \tau^k_t r_t k_t + \tau^n_t w_t n_t - \nu \tau^k_t k_t,
\]

\[
k_{t+1} = (1-\nu)k_t + i_t,
\]

\[
\log(A_{t+1}) = \mu + \rho \log(A_t) + \varepsilon_{t+1},
\]

with constraints:

\[
c_t \geq 0, \quad k_t \geq 0, \quad n_t = 1.
\]

where \( |\rho| < 1 \), \( \varepsilon_{t+1} \sim \mathcal{N}(0, \sigma^2_\varepsilon) \), and the constraints hold for \( t = 1, 2, \dots \).

\subsection*{Question 5(d)}
After gathering actual data to compare with the simulated productivity, consumption and investment per capita.

\begin{figure}[H]
    \centering
    \includegraphics[width=0.8\textwidth]{compare real vs simulated.png}
    \caption{Comparison of Simulated and Real Data}
    \label{fig:gdp}
\end{figure}

\subsection*{Log GDP per Worker Compared to Simulated Production Function}
\begin{itemize}
    \item \textbf{Blue line (Real data of GDP per worker)}: Increases steadily over time.
    \item \textbf{Red line (Simulated production function)}: Simulated data has little fluctuation and does not show a clear increase or decrease over the years.
    \item \textbf{Comparison}: The simulated line has a value much lower than the actual value, suggesting that the model may not reflect the correct production function of the worker.
\end{itemize}

\subsection*{Log Consumption Compared to Simulated Consumption}
\begin{itemize}
    \item \textbf{Blue line (Real data of Consumption)}: Increases steadily over time.
    \item \textbf{Red line (Simulated consumption function)}: Simulated data has little fluctuation and does not show a clear increase or decrease over the years.
    \item \textbf{Comparison}: The simulated line has a much lower value than the actual value, suggesting that the model may not reflect the correct consumption function per capita.
\end{itemize}

\subsection*{Log Investment per Capita Compared to Simulated Investment}
\begin{itemize}
    \item \textbf{Blue line (Real data of Investment per Capita)}: Increases steadily over time.
    \item \textbf{Red line (Simulated Investment function)}: Simulated data has little fluctuation and does not show a clear increase or decrease over the years.
    \item \textbf{Comparison}: The simulated line has a much lower value than the actual value, suggesting that the model may not reflect the correct investment function per capita.
\end{itemize}


\subsection*{Descriptive statistics table summarizing key measures: mean, median, and standard deviation}
\begin{figure}[H]
    \centering
    \includegraphics[width=0.8\textwidth]{Table real vs simu.png}
    \caption{Descriptive Statistics Table: Real vs Simulated Dat}
    \label{fig:gdp}
\end{figure}
This table shows the difference between the actual data and the simulated data. The actual values, such as \textbf{Log GDP}, \textbf{Log Consumption}, and \textbf{Log Investment per Capita}, all have higher \textbf{maximum, minimum, mean, and median} values compared to the simulated data. However, in terms of \textbf{standard deviation}, the simulated data exhibits lower values, indicating that the model is more stable than the actual data obtained.
\subsection*{Question 5(e)}
For the counterfactual analysis, we simulate \textbf{productivity, consumption, investment, capital choice, and government spending} under three different capital tax rates: 0.1, 0.2, and 0.25. Notably, 0.2 represents the initial capital tax rate in our model.

\begin{figure}[H]
    \centering
    \begin{subfigure}{0.48\textwidth}
        \centering
        \includegraphics[width=\textwidth]{simu con.png}
        \caption{Simulated Consumption}
        \label{fig:image1}
    \end{subfigure}
    \hfill
    \begin{subfigure}{0.48\textwidth}
        \centering
        \includegraphics[width=\textwidth]{simu cap.png}
        \caption{Simulated Capital Choice}
        \label{fig:image2}
    \end{subfigure}

    \begin{subfigure}{0.48\textwidth}
        \centering
        \includegraphics[width=\textwidth]{simu inv.png}
        \caption{Simulated Investment}
        \label{fig:image3}
    \end{subfigure}
    \hfill
    \begin{subfigure}{0.48\textwidth}
        \centering
        \includegraphics[width=\textwidth]{simu gs.png}
        \caption{Simulated Government Spending}
        \label{fig:image4}
    \end{subfigure}

    \begin{subfigure}{0.6\textwidth}
        \centering
        \includegraphics[width=\textwidth]{simu out.png}
        \caption{Simulated Output}
        \label{fig:image5}
    \end{subfigure}

    \caption{Comparison of Real and Simulated Data}
    \label{fig:comparison}
\end{figure}

\subsection*{Descriptive statistics table summarizing key measures: mean, median, and standard deviation for different Capital tax rate}


\begin{figure}[H]
    \centering
    \includegraphics[width=0.8\textwidth]{tab simu.png}
    \caption{Descriptive Statistics Table for different Captial tax rate}
    \label{fig:gdp}
\end{figure}

The table below shows the impact of the capital tax rate compared to the benchmark values (\(\tau_k = 0.1, 0.2, 0.25\)) on the simulated variables.

One of the highlights in this table is that \textbf{Capital Choice} decreases sharply as the capital tax rate increases. The mean declines from 5.59 for \(\tau_k = 0.1\) to 5.11 for \(\tau_k = 0.25\). This is reasonable because when capital is taxed at a higher rate, the incentive to accumulate capital diminishes, leading to a lower level of capital in the economy.

In contrast to the decline in capital choice, \textbf{Consumption per Capita} also decreases as \(\tau_k\) increases, with the mean falling from 1.2351 (\(\tau_k = 0.1\)) to 1.2285 (\(\tau_k = 0.25\)). This phenomenon can be explained by the fact that when the capital tax rate increases, households tend to reduce savings and consume more in the short run.

Finally, \textbf{Government Spending} decreases as \(\tau_k\) increases, with the mean declining from 0.2444 (\(\tau_k = 0.1\)) to 0.2372 (\(\tau_k = 0.25\)). This may be due to the reduction in tax revenue, which affects the government's spending capacity.

\newpage
\begin{center}
    \section*{References}
\end{center}

Kinh Te Va Du Bao. (2024). \textit{Vietnam's economy from 2020 to 2023 and forecast for 2024}.  

Retrieved from \url{https://kinhtevadubao.vn/kinh-te-viet-nam-giai-doan-2020-2023-va-du-bao-nam-2024-28160.html?utm_source=chatgpt.com}

\vspace{1em}

Vietstock. (2023). \textit{Total realized investment capital in 2023 is estimated at over VND 3.42 quadrillion, up 6.2\% from the previous year}.  

Retrieved from \url{https://vietstock.vn/2023/12/von-dau-tu-thuc-hien-toan-xa-hoi-nam-2023-uoc-dat-hon-342-trieu-ty-dong-tang-62-so-voi-nam-truoc-768-1137892.htm?utm_source=chatgpt.com}

\vspace{1em}

Vietnam Academy of Social Sciences (VASS). (2023). \textit{The impact of the Covid-19 pandemic on labor and employment in Vietnam}.  

Retrieved from \url{https://app.vass.gov.vn/nghien-cuu-khoa-hoc-xa-hoi-va-nhan-van/Dai-dich-Covid-19-tac-dong-den-lao-dong-viec-lam-o-Viet-Nam-126?utm_source=chatgpt.com}

\vspace{1em}

Vietnam WTO Center. (2018). \textit{Summary of Vietnam's Free Trade Agreements as of November 2018}.  

Retrieved from \url{https://trungtamwto.vn/thong-ke/12065-tong-hop-cac-fta-cua-viet-nam-tinh-den-thang-112018}

\vspace{1em}

Ministry of Planning and Investment. (2024). \textit{Investment capital from the state budget in 2023}.  

Retrieved from \url{https://mpi.gov.vn/portal/Pages/2024-1-2/Von-dau-tu-tu-nguon-ngan-sach-nha-nuoc-nam-2023-ta12fwxf.aspx?utm_source=chatgpt.com}

\vspace{1em}

Adda, J., \& Cooper, R. (2003). \textit{Dynamic Economics: Quantitative Methods and Applications}.  

The MIT Press.


\end{document}


